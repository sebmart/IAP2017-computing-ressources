\documentclass{beamer}

\mode<presentation>
{
  \usetheme{CambridgeUS}
  \usecolortheme{beaver}
  \usefonttheme{default}
  \setbeamertemplate{navigation symbols}{}
  \setbeamertemplate{caption}[numbered]
}

\usepackage[english]{babel}
\usepackage[utf8x]{inputenc}
\usepackage{color}
\usepackage{minted}
\usemintedstyle{native}

\title[IAP-2017]{Using Computational Resources in Optimization and Statistics}
\author{Sébastien Martin}
\institute{MIT}
\date{Tuesday 01/24/2017}

\begin{document}

\begin{frame}
  \titlepage
\end{frame}


\section{Motivation}

\begin{frame}{Heavy Computations}
  In Optimization and Statistics, we often need a lot of computational power:
  \begin{itemize}
    \item Machine Learning on large datasets
    \item Hard optimization problems, mixed integer programming
  \end{itemize}
  \pause
  Or repetitive computations:
  \begin{itemize}
    \item Parameter tuning
    \item Benchmarking
  \end{itemize}
\end{frame}

\begin{frame}{Limitations of a Personal Computer}
  Using your personal computer may seem simple, but there are serious limitations:
  \begin{itemize}
    \item<1-> Limited \alert{memory} (Big Data, large matrices...)
    \item<2-> Limited \alert{computational power}.
    \item<3-> Limited \alert{number of machines}/cores.
    \item<4-> Limited \alert{time} available. (you want to use your laptop for other things too..)
  \end{itemize}
\end{frame}

\begin{frame}{How does it work?}
    diagrams : jobs vs interactive
\end{frame}

\begin{frame}{Resources}
  Engaging, AWS, Athena, another personal computer...
\end{frame}

\begin{frame}{Why Should I Use this}
  - research breakthrough: try more parameter, run overnight
  - Can be very simple (R-Studio...)
  - Using Bigger datasets
  - Useful skill in general, at the age of cloud
  - learn how to use console (and GitHub), for real.
\end{frame}
\end{document}
